 
\documentclass[11pt,a4paper]{article}
\usepackage[utf8]{inputenc}		% LaTeX, comprend les accents !
\usepackage[T1]{fontenc}
\usepackage{natbib}	
%\usepackage[square,sort&compress,sectionbib]{natbib}		% Doit être chargé avant babel      
\usepackage[frenchb,english]{babel}
\usepackage{lmodern}
\usepackage{amsmath,amssymb, amsthm}
\usepackage{a4wide}
\usepackage[capposition=top]{floatrow}
\usepackage{verbatim}
\usepackage{float}
\usepackage{placeins}
\usepackage{flafter}
\usepackage{longtable}
\usepackage{import}
\usepackage{pdflscape}
\usepackage{rotating}
\usepackage{hhline}
\usepackage{multirow}
\usepackage{booktabs}
\usepackage[pdftex,pdfborder={0 0 0},colorlinks=true,linkcolor=blue,urlcolor=blue,citecolor=blue,bookmarksopen=true]{hyperref}
\usepackage{eurosym}
%\usepackage{breakcites}
\usepackage[autostyle]{csquotes}
%\usepackage{datetime}
\usepackage{natbib}
\usepackage{setspace}
\usepackage{lscape}
\usepackage[usenames]{color}
\usepackage{indentfirst}
\usepackage{forest}
\usepackage{url}
\usepackage{enumitem}
\usepackage{multirow}
\usepackage{subcaption}
\usepackage[justification=centering]{caption}
\bibliographystyle{agsm}

\usepackage{array}

\newcommand{\isEmbedded}{true}

\graphicspath{{../bordeaux/results/}}


\begin{document}

\selectlanguage{frenchb}
\title{Document préparatoire \\ Séance de travail des 5 et 6 octobre 2016}


\author{Sophie Cottet, Mahdi Ben Jelloul et Simon Rabat\'e}


\maketitle

% Introduction
Cette note propose des premières pistes d'estimation pour la modélisation de l'évolution des rémunérations à partir des grilles indiciaires dans les fonctions publiques territoriales et hospitalières. 
Dans un premier temps, nous revenons sur les processus que l'on souhaite modéliser. L'évolution de la rémunération des fonctionnaires dépend des changements de grades et des changements d'échelon au sein des grades. Nous cherchons ensuite à documenter ces processus dans les données transmises par la CDC. Le manque de profondeur temporel dans les données limite pour l'instant l'analyse des données. En conséquence, les suggestions de modélisation qui en découlent restent globalement spéculatives; les approches envisagées à ce stade sont toutefois décrites dans la troisième partie de ce document. Enfin, dans une dernière partie nous décrivons des méthodes de simulation à partir des estimations, pouvant servir de base aux tests d'adéquation des modèles économétriques voire à la simulations des trajectoires dans le modèle de microsimulation. 

% Section I: principe général 
\ifx\isEmbedded\undefined


\documentclass[11pt,a4paper]{article}
\usepackage[utf8]{inputenc}		% LaTeX, comprend les accents !
\usepackage[T1]{fontenc}
\usepackage{natbib}	
%\usepackage[square,sort&compress,sectionbib]{natbib}		% Doit être chargé avant babel      
\usepackage[frenchb,english]{babel}
\usepackage{lmodern}
\usepackage{amsmath,amssymb, amsthm}
\usepackage{a4wide}
\usepackage[capposition=top]{floatrow}
\usepackage{verbatim}
\usepackage{float}
\usepackage{placeins}
\usepackage{flafter}
\usepackage{longtable}
\usepackage{pdflscape}
\usepackage{rotating}
\usepackage{hhline}
\usepackage{multirow}
\usepackage{booktabs}
\usepackage[pdftex,pdfborder={0 0 0},colorlinks=true,linkcolor=blue,urlcolor=blue,citecolor=blue,bookmarksopen=true]{hyperref}
\usepackage{eurosym}
\usepackage{breakcites}
\usepackage[autostyle]{csquotes}
%\usepackage{datetime}
\usepackage{natbib}
\usepackage{setspace}
\usepackage{lscape}
\usepackage[usenames]{color}
\usepackage{indentfirst}

\usepackage{url}
\usepackage{enumitem}
\usepackage{multirow}
\usepackage{subcaption}
\usepackage[justification=centering]{caption}
\bibliographystyle{agsm}

\usepackage{array}

\begin{document}
\selectlanguage{frenchb}
\else \fi
%%%%%%%%%%%%%%%%%%%%%%%%%%%%%%%%%%%%%%%%%%%%%%%%%%%%%%%%%%%%%%%%%%%%%%%%%%%%%%%%%%%%%%%%%%%%%%%%%%%%%%%%%%%%%%



\section{L'objectif: modéliser les rémunérations à partir des grilles}


\subsection*{Retour sur la classification des emplois}

L'organisation de la carrière d'un fonctionnaire est fondée sur une grille de classements des emplois, avec différents niveaux:
 
\begin{enumerate}[leftmargin=1cm ,parsep=0cm,itemsep=0cm,topsep=0cm] 
\item Les cadres ou corps d'emploi,
\item Les filières: regroupements informels des corps d'emploi (10 dans la FPT, 6 dans la FPH), 
\item Les catégories hiérarchiques: les fonctionnaires peuvent être de catégorie A, B ou C,
\item Les grades: chaque corps d'emploi est segmenté en un ou plusieurs grades, régulés par un statut particulier,
\item Les échelons définissant le niveau de l'indice brut au sein du grade. 
\end{enumerate}

% TODO: j'inverserai les 3 premiers pour les mettre dans l'ordre 2

\vspace{0.5cm}

La grille de rémunération est définie pour chaque grade: elle donne le niveau de l'indice brut pour un échelon donné. 


Points à préciser: 
\begin{itemize}[leftmargin=1cm ,parsep=0cm,itemsep=0cm,topsep=0cm] 
\item Y a-t-il bien une délimitation nette entre cadres/filières et catégories hiérarchiques? (la filière X n'est composée que de fonctionnaires de catégorie A). 
\item Quelles conditions de passage d'un grade à l'autre au sein d'un corps? Passage automatique comme pour les échelons ou plus discrétionnaire? %Question à supprimer, cf. mail Sophie
\end{itemize}


\subsection*{Les phénomènes à modéliser}

Si on met de côté pour l'instant le taux de prime, la modélisation du salaire des fonctionnaires dépend de l'évolution de la rémunération, elle-même définie directement par l'évolution de l'indice brut. 

Modéliser l'évolution de l'indice revient donc à modéliser deux phénomènes principaux de la carrière d'un individu: 
\begin{itemize}[leftmargin=1cm ,parsep=0cm,itemsep=0cm,topsep=0cm] 
\item La progression au sein d'un grade, c'est-à-dire la vitesse à laquelle les échelons sont franchis,
\item Les changements de grade, qui regroupent en fait deux questions: (i) à quelle fréquence les individus changent-il de grade et (ii) vers où vont-ils? 
\end{itemize} 

\vspace{0.2cm}



La question centrale de la modélisation de l'évolution est donc la suivante: pour quel type de phénomène a-t-on de la variabilité inter-individuelle? Plus les grilles sont rigides, plus la modélisation choisie peut-être simple. A l'extrême, si la durée dans chaque échelon est fixe et que le changement de grade suit une règle fixe (par exemple, "tous les individus arrivés au bout du grade G1 passent au grade G2"), l'évolution de la rémunération dépend directement de l'évolution des grilles et ne nécessite pas de travail de modélisation. La modélisation est nécessaire car, en réalité, la carrière des individus ne suit pas un chemin prédéfini. L'enjeu principal est donc la modélisation de la déviation par rapport à ce "tapis roulant". Cette déviation peut intervenir au niveau de la durée passée dans chaque échelon, au niveau du moment où intervient le changement de grade (avant la fin de la grille ou en fin de grille), et au niveau du grade de destination après le changement. 

Il s'agit d'une question en partie législative: dans quelle mesure est fixe la durée passée dans l'échelon (durée minimale, durée maximale, ou durée fixe), et dans quelle mesure le passage d'un grade à l'autre est automatique au sein d'un corps (condition de promotions: concours, ou simplement âge ou durée dans le grade ?). A rigidité législative donnée, il s'agit d'une question empirique: quelle variance observe-t-on dans la durée passée dans chaque échelon au sein d'un grade ? Quelle proportion d'individus est promue au sein de son corps dans le grade supérieur? A quel moment ces promotions se produisent? Quelle proportion d'individu change de grade sans passer dans le grade immédiatement supérieur (changement de corps, de catégorie, de fonction publique) ?

Dans la partie suivante, nous tentons de documenter ces questions à partir des données disponibles à ce stade. 



\vspace{0.5cm}
Points à préciser: 
\begin{itemize}[leftmargin=1cm ,parsep=0cm,itemsep=0cm,topsep=0cm] 
\item Quelle interaction entre le module \og rémunération \fg\ et les modules \og carrière \fg\ et \og affiliation \fg ?  
\item Comment différencier un changement de grade d'une sortie de la FP ou d'une disponibilité, sachant que ces deux phénomènes ne sont pas forcément décorrélés (un individus peut démissionner plus facilement de la FPT-FPH s'il ne peut pas accéder à un grade supérieur) ? 
\end{itemize}









%%%%%%%%%%%%%%%%%%%%%%%%%%%%%%%%%%%%%%%%%%%%%%%%%%%%%%%%%%%%%%%%%%%%%%%%%%%%

\ifx\isEmbedded\undefined
\newpage
\bibliographystyle{../../Divers/myagsm} 
\bibliography{../../Divers/biblio_these}
\end{document}
\else \fi




% Section II: data
\ifx\isEmbedded\undefined


\documentclass[11pt,a4paper]{article}
\usepackage[utf8]{inputenc}		% LaTeX, comprend les accents !
\usepackage[T1]{fontenc}
\usepackage{natbib}	
%\usepackage[square,sort&compress,sectionbib]{natbib}		% Doit être chargé avant babel      
\usepackage[frenchb,english]{babel}
\usepackage{lmodern}
\usepackage{amsmath,amssymb, amsthm}
\usepackage{a4wide}
\usepackage[capposition=top]{floatrow}
\usepackage{verbatim}
\usepackage{float}
\usepackage{placeins}
\usepackage{flafter}
\usepackage{longtable}
\usepackage{pdflscape}
\usepackage{rotating}
\usepackage{hhline}
\usepackage{multirow}
\usepackage{booktabs}
\usepackage[pdftex,pdfborder={0 0 0},colorlinks=true,linkcolor=blue,urlcolor=blue,citecolor=blue,bookmarksopen=true]{hyperref}
\usepackage{eurosym}
\usepackage{breakcites}
\usepackage[autostyle]{csquotes}
%\usepackage{datetime}
\usepackage{natbib}
\usepackage{setspace}
\usepackage{lscape}
\usepackage[usenames]{color}
\usepackage{indentfirst}

\usepackage{url}
\usepackage{enumitem}
\usepackage{multirow}
\usepackage{subcaption}
\usepackage[justification=centering]{caption}
\bibliographystyle{agsm}

\usepackage{array}

\begin{document}

\else \fi
%%%%%%%%%%%%%%%%%%%%%%%%%%%%%%%%%%%%%%%%%%%%%%%%%%%%%%%%%%%%%%%%%%%%%%%%%%%%%%%%%%%%%%%%%%%%%%%%%%%%%%%%%%%%%%



\section{L'analyse des données: résultats préliminaires}

En amont du choix de modélisation, il est nécessaire de documenter la variabilité individuelle dans les différents phénomènes que l'on souhaite modéliser. Par rapport à des trajectoires totalement déterministes, l'aléa peut venir (i) de la durée passée dans l'échelon (ii) de la possibilité de changement de grade en milieu de grille et (iii) le grade de destination quand on observe un changement de grade. 

\subsection{Analyser la vitesse de progression: un travail de complétion nécessaire}

Nous regroupons les points (i) et (ii) autour de la notion de vitesse de progression de carrière: vitesse de progression dans l'échelon et d'un grade à l'autre. Nous voudrions comparer la durée passée dans un échelon ou un grade donné pour un individu donné, à la vitesse prévue par la grille d'une part, et à la durée observé pour le reste de la population d'autre part. Cela nécessite d'être en mesure de connaitre la durée passée dans le grade ou l'échelon à chaque date donnée, et donc un certain recul historique. 

Comme mentionné plus haut, la reconstitution de la trajectoire statutaire -- c'est-à-dire de l'échelon et du grade -- sur le passé (avant 2011) reste à affiner. Dès lors nous n'avons pu à ce stade documenter de manière précise ces éléments, pourtant centraux pour la modélisation de l'évolution de la rémunération. 


\subsection{Analyser les grades de destination: premiers résultats}

L'analyse des données sur les années 2011-2014 permet de documenter une autre question importante par rapport aux choix de modélisation: le grade de destination quand on observe un changement de grade pour un individu donné. L'approche adoptée est la suivante: pour chaque grade, nous considérons les années pour laquelle nous observons un changement de grade entre l'année $n$ et l'année $n+1$, et calculons les agrégats suivantes: 


\begin{itemize}[leftmargin=1cm ,parsep=0cm,itemsep=0cm,topsep=0cm] 
\item Le nombre de grades possibles en $n+1$
\item La proportion d'individus passant dans le grade le plus représenté en $n+1$, les deux plus représentés, les trois plus représentés, et les 5 plus représentés. 
\end{itemize}

Les moyennes sur l'ensemble de la population, pondérée ou non en fonction du nombre de transitions observées pour le grade considérée, sont présentées à la table \ref{means}. Même s'il existe un nombre élevé de transitions possibles en moyenne, la grande majorité des transitions se fait vers un petit nombres de grade possible. Ainsi par exemple 94\% des transitions observés se font vers les 3 destinations principales (propres à chaque grade). Ce constat est confirmé au graphique \ref{pct},

\begin{table}[ht]
\label{means}
\centering
\caption{Destinations en cas de changement de grade} 
\begin{tabular}{l|cc}
  \hline
 & Moyenne simple & Moyenne pondérée \\ 
  \hline
Nombre de destinations & 11.4 & 45.5 \\ 
  Part de la destination majoritaire   & 66.1 \% & 58.8 \%  \\ 
  Part des 2 destinations majoritaires & 87.4 \% & 87.0 \% \\ 
  Part des 3 destinations majoritaires & 93.6 \% & 92.8 \% \\ 
  Part des 5 destinations majoritaires & 97.5 \% & 96.1 \% \\ 
   \hline
\end{tabular}
\end{table}

\begin{figure}[t]
  \label{pct}
\caption{Distribution de la proportion de transitions vers les destinations principales}
\vspace{-0.1cm}
\centering
  \includegraphics[width=0.7\linewidth]{../../../fonction-publique/fonction_publique/bordeaux/results/pct.pdf}
\vspace{0.1cm}  
\begin{minipage}{12cm}%
\small \textsc{Lecture:} Pour environ 50 \% des grades, les 5 destinations les plus fréquentes représentent 100 \% des transitions observées.  
 \end{minipage}%
\end{figure}

En se concentrant sur les grades les plus représentés\footnote{TTH1 adjoint technique de 2eme classe, TAJ1 adjoint administratif de 2ème classe, 3001 aide soignant classe normale, 2432 infirmier de classe normale}
pour les générations 1970-1979, il apparaît (voir la table \ref{tab:destination}) que l'écrasante majorité des transitions se fait dans le garde immédiatement supérieur quand elles ne se font pas vers un non-grade.
\begin{table}[htbp]
    \label{tab:destination}
    \centering
    \caption{Destinations en cas de changement de grade (avec grade vide)} 
    \begin{tabular}{llrr}
\toprule
initial & destination &  nombre &   part \\
\midrule
        &      autres &   90946 & 0.68 \% \\
        &        TTH1 &   21439 & 0.16 \% \\
        &        3001 &   10797 & 0.08 \% \\
        &        TAJ1 &   10137 & 0.08 \% \\
   TTH1 &             &    8416 & 0.43 \% \\
   TTH1 &        TTH2 &    8231 & 0.42 \% \\
   TTH1 &      autres &    2490 & 0.13 \% \\
   TTH1 &        TMD1 &     431 & 0.02 \% \\
   3001 &             &    5900 & 0.47 \% \\
   3001 &        3002 &    5389 & 0.43 \% \\
   3001 &      autres &     795 & 0.06 \% \\
   3001 &        2432 &     533 & 0.04 \% \\
   TAJ1 &        TAJ2 &    6346 & 0.48 \% \\
   TAJ1 &             &    5511 & 0.42 \% \\
   TAJ1 &      autres &     878 & 0.07 \% \\
   TAJ1 &        TAR1 &     396 & 0.03 \% \\
\bottomrule
\end{tabular}

\end{table}

En se limitant aux carrières ne présentant pas d'épisode ou la grade n'est pas renseigné, il apparaît que la destination est dans près de 80 \% des cas le grade immédiatement supérieur (passage à la classe supérieure) dans le corps ou le cadre d'emploi (voir la table \ref{tab:destination}).       

\begin{table}[htbp]
    \label{tab:purged_destination}
    \centering
    \caption{Destinations en cas de changement de grade (carrières sans grade vide)} 
    \begin{tabular}{llrr}
\toprule
initial & destination &  nombre &   part \\
\midrule
   TTH1 &        TTH2 &    7479 & 0.75 \% \\
   TTH1 &      autres &    1736 & 0.17 \% \\
   TTH1 &        TMD1 &     383 & 0.04 \% \\
   TTH1 &        TAJ1 &     362 & 0.04 \% \\
   3001 &        3002 &    5207 & 0.82 \% \\
   3001 &        2432 &     510 & 0.08 \% \\
   3001 &      autres &     505 & 0.08 \% \\
   3001 &        3121 &     134 & 0.02 \% \\
   TAJ1 &        TAJ2 &    5581 & 0.84 \% \\
   TAJ1 &      autres &     446 & 0.07 \% \\
   TAJ1 &        TAR1 &     345 & 0.05 \% \\
   TAJ1 &        TTH1 &     278 & 0.04 \% \\
   2432 &        2753 &    2645 & 0.79 \% \\
   2432 &      autres &     416 & 0.12 \% \\
   2432 &        2801 &     188 & 0.06 \% \\
   2432 &        1801 &      82 & 0.02 \% \\
\bottomrule
\end{tabular}

\end{table}

Questions à discuter
\begin{enumerate}[leftmargin=1cm ,parsep=0cm,itemsep=0cm,topsep=0cm] 
\item Quel statut des transitions vers une missing value? Problème de donnée ou disponibilité ou sortie de la FP? Module rémunération, carrière ou affiliation? 
\end{enumerate}



%%%%%%%%%%%%%%%%%%%%%%%%%%%%%%%%%%%%%%%%%%%%%%%%%%%%%%%%%%%%%%%%%%%%%%%%%%%%

\ifx\isEmbedded\undefined
\newpage
\bibliographystyle{../../Divers/myagsm} 
\bibliography{../../Divers/biblio_these}
\end{document}
\else \fi





% Section III: Modélisation
\ifx\isEmbedded\undefined


\documentclass[11pt,a4paper]{article}
\usepackage[utf8]{inputenc}		% LaTeX, comprend les accents !
\usepackage[T1]{fontenc}
\usepackage{natbib}	
%\usepackage[square,sort&compress,sectionbib]{natbib}		% Doit être chargé avant babel      
\usepackage[frenchb,english]{babel}
\usepackage{lmodern}
\usepackage{amsmath,amssymb, amsthm}
\usepackage{a4wide}
\usepackage[capposition=top]{floatrow}
\usepackage{verbatim}
\usepackage{float}
\usepackage{placeins}
\usepackage{flafter}
\usepackage{longtable}
\usepackage{pdflscape}
\usepackage{rotating}
\usepackage{hhline}
\usepackage{multirow}
\usepackage{booktabs}
\usepackage[pdftex,pdfborder={0 0 0},colorlinks=true,linkcolor=blue,urlcolor=blue,citecolor=blue,bookmarksopen=true]{hyperref}
\usepackage{eurosym}
\usepackage{breakcites}
\usepackage[autostyle]{csquotes}
%\usepackage{datetime}
\usepackage{natbib}
\usepackage{setspace}
\usepackage{lscape}
\usepackage[usenames]{color}
\usepackage{indentfirst}

\usepackage{url}
\usepackage{enumitem}
\usepackage{multirow}
\usepackage{subcaption}
\usepackage[justification=centering]{caption}
\bibliographystyle{agsm}

\usepackage{array}

\begin{document}

\else \fi
%%%%%%%%%%%%%%%%%%%%%%%%%%%%%%%%%%%%%%%%%%%%%%%%%%%%%%%%%%%%%%%%%%%%%%%%%%%%%%%%%%%%%%%%%%%%%%%%%%%%%%%%%%%%%%



\section{Modélisation économétrique}



\subsection{Modéliser la vitesse dans le grade}


\subsection{Modéliser le \og choix \fg{} en fin de grille}


\subsection{Modéliser les changement de grade en cours de grille}



\subsection{Les changements de grade: logit multinomial}

Nous avons vu %à vérifier :)
que le classement hiérarchique des agents de la fonction publique n'est pas exactement un \og tapis roulant \fg, et qu'entre autres, lorsqu'un agent arrive en fin de grade, 
le grade de \og destination \fg\ varie. Pour pouvoir projeter les carrières des agents, il faut donc estimer la probabilité de passer dans tel ou tel grade après avoir fait
une carrière dans tel autre grade.

Cette problématique évoque une modélisation du phénomène comme \og choix discrets \fg\ : il s'agirait d'estimer la probabilité d'un ensemble de choix, non ordonnés.
C'est une modélisation répandue notamment en économie du travail, lorsqu'on cherche à estimer la probabilité d'être en emploi à temps plein, en emploi à temps partiel,
en recherche d'emploi, actif inoccupé, \textit{etc.} L'estimation se fait par logit multinomial.

Une limite importante de ce type de modélisation est que, le logit multinomial étant une généralisation du modèle binaire, on suppose que le choix entre deux alternatives
(par exemple, passer dans le grade X ou Y après avoir été dans le grade V) est indépendant des autres alternatives (le passage au grade Z après avoir été dans le grade V).
Cette hypothèse ne semble pas problématique dans le cas de l'exemple donné, mais si on souhaite aussi modéliser le congé maladie, la mise en disponibilité ou le détachement,
l'hypothèse est moins crédible.% : ce qui rend la mise en disponibilité plus attractive rend probablement moins souhaitable de passer dans le grade X, Y ou Z.
% Non c'est pas ça. Expliquer pourquoi


Autre contrainte, la somme des probabilités associées aux différentes alternatives doit être égale à 1 (les différents choix doivent décrire l'univers des possibles). Cela impliquera sans doute d'avoir une possibilité de choix résiduelle \og autres \fg{}, en plus des principales destinations. 

-> Logits emboîtés

+ Mixed logit : pour intégrer des coefficients pour certains choix mais pas pour d'autres. (cf. cours de Luc)


\subsection{Le parcours dans le grade: un modèle dynamique ?}





Question: quel degré de finesse dans le choix des sous-catégories? 
Par FP, par corps? par catégorie hiérarchique? 



%%%%%%%%%%%%%%%%%%%%%%%%%%%%%%%%%%%%%%%%%%%%%%%%%%%%%%%%%%%%%%%%%%%%%%%%%%%%

\ifx\isEmbedded\undefined
\newpage
\bibliographystyle{../../Divers/myagsm} 
\bibliography{../../Divers/biblio_these}
\end{document}
\else \fi





% Section IV: Simulation
\ifx\isEmbedded\undefined


\documentclass[11pt,a4paper]{article}
\usepackage[utf8]{inputenc}		% LaTeX, comprend les accents !
\usepackage[T1]{fontenc}
\usepackage{natbib}	
%\usepackage[square,sort&compress,sectionbib]{natbib}		% Doit être chargé avant babel      
\usepackage[frenchb,english]{babel}
\usepackage{lmodern}
\usepackage{amsmath,amssymb, amsthm}
\usepackage{a4wide}
\usepackage[capposition=top]{floatrow}
\usepackage{verbatim}
\usepackage{float}
\usepackage{placeins}
\usepackage{flafter}
\usepackage{longtable}
\usepackage{pdflscape}
\usepackage{rotating}
\usepackage{hhline}
\usepackage{multirow}
\usepackage{booktabs}
\usepackage[pdftex,pdfborder={0 0 0},colorlinks=true,linkcolor=blue,urlcolor=blue,citecolor=blue,bookmarksopen=true]{hyperref}
\usepackage{eurosym}
\usepackage{breakcites}
\usepackage[autostyle]{csquotes}
%\usepackage{datetime}
\usepackage{natbib}
\usepackage{setspace}
\usepackage{lscape}
\usepackage[usenames]{color}
\usepackage{indentfirst}

\usepackage{url}
\usepackage{enumitem}
\usepackage{multirow}
\usepackage{subcaption}
\usepackage[justification=centering]{caption}
\bibliographystyle{agsm}

\usepackage{array}

\begin{document}

\else \fi
%%%%%%%%%%%%%%%%%%%%%%%%%%%%%%%%%%%%%%%%%%%%%%%%%%%%%%%%%%%%%%%%%%%%%%%%%%%%%%%%%%%%%%%%%%%%%%%%%%%%%%%%%%%%%%



\section{Microsimulation}


Afin de valider la performance modélisation retenue et les résultats de l'estimation économétrique, il nous semble utile de pouvoir réaliser une simulation rétrospective. En partant d'un état initial plus ou moins ancien et en faisant évoluer individuellement les agents, nous espérons être en mesure de détecter les limites des différentes modélisations pour les améliorer itérativement. A cette fin, il nous faut pouvoir microsimuler de façon efficace (temps de calcul, gestion de la mémoire) l'évolution de la population initiale.  


\subsection{Le parcours dans le grade}

Nous avons réalisé un prototype de parcours dans le grade à une vitesse de passage d'échelon donnée tout en respectant la législation et notamment les changements de grilles au cours du temps. Nous avons tenté de vectoriser au maximum le programme pour que l'exécution soit la plus rapide possible.
Les boucles se font donc sur les grades représentés et si nécessaires les échelons représentés. L'algorithme consiste à appliquer successivement les opérations suivantes:
\begin{itemize}
	\item Renseigner l'état initial des individus (date de l'observation, grade, échelon)
	\item Déterminer la date d'effet de la grille en cours et de la suivante
	\item Calcul de la durée dans l'échelon selon la grille en effet à l'état initial (et la vitesse de parcours de l'échelon le cas échéant)
	\item Calcul de la date d'effet d'une éventuelle grille réformée avant la fin de l'échelon
	\item Calcul de la durée dans l'échelon avec cette nouvelle grille
	\item Calcul de la durée effective dans l'échelon
	\item Calcul de la date de fin dans l'échelon
\end{itemize}
 

\subsection{Les changements de grade}

Si le modèle économétrique permets d'identifier les dates de changements de grade en cours de grade ou les modalités de promotion au grade supérieur, il devrait être possible de le coupler à l'algorithme précédent pour obtenir un algorithme permettant de simuler une carrière complète.





%%%%%%%%%%%%%%%%%%%%%%%%%%%%%%%%%%%%%%%%%%%%%%%%%%%%%%%%%%%%%%%%%%%%%%%%%%%%

\ifx\isEmbedded\undefined
\newpage
\bibliographystyle{../../Divers/myagsm} 
\bibliography{../../Divers/biblio_these}
\end{document}
\else \fi



\end{document}


